\documentclass[handout]{beamer}

\usepackage[T1]{fontenc}
\usepackage{lmodern}
\usepackage{soul}
\usepackage{url}

% Theme.
\usetheme{Bergen}
\usecolortheme{rose}
\urlstyle{same}
\setbeamertemplate{navigation symbols}{}

% Table spacing.
\renewcommand{\arraystretch}{1.2}
\setlength{\tabcolsep}{1em}

% About.
\title[XML]{XML, XSD, XJC, JAXB \& REST}
\author[Jasper A. Visser]{Jasper A. Visser}
\institute[Portavita]{Portavita BV}
\titlegraphic{\includegraphics[height=15mm]{javaxml-duke.png}}
\logo{\includegraphics[height=10mm]{logo.jpg}\hspace{3mm}}
\date{Februari 26\textsuperscript{th}, 2013}

\begin{document}

\begin{frame}
	\inserttitlegraphic
	\vspace{5mm}
	\titlepage
\end{frame}

\begin{frame}
	\frametitle{Outline}
	\tableofcontents
\end{frame}

%%%%% About %%%%%%%%%%%%%%%%%%%%%%%%%%%%%%%%%%%%%%%%%%%%%%%%%%%%%%%%%%%%%%%%%%%
\section*{About}
\begin{frame}
	\frametitle{\insertsection}
	\begin{block}{Goals}
		\begin{itemize}
			\item XML namespaces
			\item XML schemas
			\item Parse \& build XML in Java
			\item Build REST webservice
		\end{itemize}
	\end{block}
	\begin{semiverbatim}
		Show sample.xml
	\end{semiverbatim}
\end{frame}

%%%%% XML %%%%%%%%%%%%%%%%%%%%%%%%%%%%%%%%%%%%%%%%%%%%%%%%%%%%%%%%%%%%%%%%%%%%%
\section{XML}
\begin{frame}
	\frametitle{\insertsection}
	\framesubtitle{Namespaces 1/3}
	\begin{block}{Consists of}
		\begin{itemize}
			\item URL $\rightarrow$ http://foo.org/
			\item (optional) Prefix $\rightarrow$ foo\textbf{:}
			\item (optional) Schema $\rightarrow$ foo.xsd
		\end{itemize}
	\end{block}
	\begin{block}{Usage} \small
		\textbf{xmlns}="http://foo.org/" \\
		\textbf{xmlns:xsi}\\="\url{http://www.w3.org/2001/XMLSchema-instance}" \\
		\textbf{xsi:schemaLocation}="http://foo.org/ foo.xsd" \\
	\end{block}
	\begin{block}{QName} \small
		\{ "\url{http://www.w3.org/2001/XMLSchema-instance}", "schemaLocation" \} \\
	\end{block}
\end{frame}

\begin{frame}
	\frametitle{\insertsection}
	\framesubtitle{Namespaces 2/3}
	\begin{block}{Purpose}
		\begin{itemize}
			\item Compartimentalize
			\item Uniquely identify elements
			\item Validate XML instance
		\end{itemize}
	\end{block}
	\begin{example}
		$<$soap:Envelope$>$ \\
			\hspace{5mm} $<$soap:Header$>$ \\
				\hspace{10mm} \ldots \\
			\hspace{5mm} $<$/soap:Header$>$ \\
			\hspace{5mm} $<$soap:Body$>$ \\
				\hspace{10mm} \textbf{$<$foo:bar baz="qux"$>$} \\
					\hspace{15mm} \textbf{\ldots} \\
				\hspace{10mm} \textbf{$<$/foo:bar$>$} \\
			\hspace{5mm} $<$/soap:Body$>$ \\
		$<$/soap:Envelope$>$
	\end{example}
\end{frame}

\begin{frame}
	\frametitle{\insertsection}
	\framesubtitle{Namespaces 3/3}
	\begin{example}
		$<$xsl:for-each select="//person"$>$ \\
			\hspace{5mm} \textbf{$<$h1$>$} \\
				\hspace{10mm} $<$xsl:value-of select="@name"$>$ \\
			\hspace{5mm} \textbf{$<$/h1$>$} \\
		$<$/xsl:for-each$>$ \\
	\end{example}
	\begin{example}
		$<$foo:bar \textbf{xsi:type}="Baz"$>$ \\
	\end{example}
	\begin{semiverbatim}
		Add namespaces to sample.xml
	\end{semiverbatim}
\end{frame}

%%%%% XSD %%%%%%%%%%%%%%%%%%%%%%%%%%%%%%%%%%%%%%%%%%%%%%%%%%%%%%%%%%%%%%%%%%%%%
\section{XSD}
\begin{frame}
	\frametitle{\insertsection}
	\framesubtitle{Define XML Schema 1/2}
	Constrain instance document
	\begin{block}{Consists of}
		\begin{itemize}
			\item Root element
			\item Complex types $\rightarrow$ element \& attribute declarations
			\item Simple types
		\end{itemize}
	\end{block}
	\begin{block}{Root element}
		xs:element
	\end{block}
	\begin{block}{Type}
		xs:simpleType,
		xs:complexType
	\end{block}
	\begin{block}{Child elements}
		xs:sequence,
		xs:choice,
		minOccurs,
		maxOccurs
	\end{block}
	\begin{block}{Attributes}
		xs:string, use="required"
	\end{block}
	\begin{semiverbatim}
		Create sample.xsd
	\end{semiverbatim}
\end{frame}

\begin{frame}
	\frametitle{\insertsection}
	\framesubtitle{Define XML Schema 2/2}
	\begin{block}{Enum}
		xs:simpleType,
		xs:restriction,
		xs:enumeration
	\end{block}
	\begin{block}{Inheritance}
		xs:complexContent,
		xs:extension,
		xsi:type
	\end{block}
	\begin{block}{Comments}
		xs:annotation
	\end{block}
	\begin{semiverbatim}
		Update sample.xsd, sample.xml
	\end{semiverbatim}
\end{frame}

%%%%% XJC %%%%%%%%%%%%%%%%%%%%%%%%%%%%%%%%%%%%%%%%%%%%%%%%%%%%%%%%%%%%%%%%%%%%%
\section{XJC}
\begin{frame}
	\frametitle{\insertsection}
	Bundled with JDK \\
	\begin{example}
		\texttt{xjc -d src sample.xsd}
	\end{example}
	\begin{block}{Result}
		\begin{itemize}
			\item http://portavita.eu/cda $\rightarrow$ package \textbf{eu.portavita.cda} \\
			\item \emph{xs:complexType} Observation $\rightarrow$ \textbf{class Observation} \\
			\item \emph{xs:simpleType} ClassCode $\rightarrow$ \textbf{enum ClassCode} \\
			\item \emph{xs:element} code $\rightarrow$ \textbf{get-/setCode()} \\
			\item \emph{xs:attribute} moodCode $\rightarrow$ \textbf{get-/setMoodCode()} \\
		\end{itemize}
	\end{block}
	\begin{block}{Vice versa}
		\texttt{schemagen Document.java}
	\end{block}
	\begin{semiverbatim}
		Compile sample.xsd; show result
	\end{semiverbatim}
\end{frame}

%%%%% JAXB %%%%%%%%%%%%%%%%%%%%%%%%%%%%%%%%%%%%%%%%%%%%%%%%%%%%%%%%%%%%%%%%%%%%
\section{JAXB}
\begin{frame}
	\frametitle{\insertsection}
	\framesubtitle{Annotation}
	Part of Java SE, EE
	\begin{block}{Class}
		@\textbf{XmlAccessorType}(XmlAccessType.FIELD) \\
		@\textbf{XmlType}(name = "Observation") \\
		@\textbf{XmlRootElement}(name = "ClinicalDocument") \\
	\end{block}
	\begin{block}{Members}
		@\textbf{XmlElement}(required = true) \\
		@\textbf{XmlAttribute}(name = "classCode", required = true) \\
	\end{block}
\end{frame}

\begin{frame}
	\frametitle{\insertsection}
	\framesubtitle{Marshal, unmarshal}
	\begin{definition}
		\begin{itemize}
			\item Unmarshal: XML $\rightarrow$ Objects
			\item Marshal: Objects $\rightarrow$ XML
		\end{itemize}
	\end{definition}
	\begin{semiverbatim}
		Unmarshal sample.xml
	\end{semiverbatim}
	\begin{semiverbatim}
		Create \& marshal ClinicalDocument
	\end{semiverbatim}
	\begin{semiverbatim}
		Marshal 10,000 documents
	\end{semiverbatim}
	\begin{semiverbatim}
		Use eclipselink JAXB Context; marshal JSON
	\end{semiverbatim}
	% TODO: Read about JAXBContext (what is it?)
\end{frame}

%%%%% REST %%%%%%%%%%%%%%%%%%%%%%%%%%%%%%%%%%%%%%%%%%%%%%%%%%%%%%%%%%%%%%%%%%%%
\section{REST}
\begin{frame}
	\frametitle{\insertsection}
	\framesubtitle{Paradigm}
	\begin{block}{Model}
		Everything is a resource \\
		Resources are part of collections
	\end{block}
	\begin{block}{CRUD}
		\begin{itemize}
			\item Create
			\item Retreive
			\item Update
			\item Delete
		\end{itemize}
	\end{block}
	\begin{block}{Hyperlinked}
		Embed links in result sets
	\end{block}
\end{frame}

\begin{frame}
	\frametitle{\insertsection}
	\framesubtitle{Implementation}
	\begin{block}{Protocol}
		HTTP
	\end{block}
	\begin{block}{URL}
		\textbf{Collection} \\
		\url{http://localhost:7777/document/} \\
		\textbf{Element} \\
		\url{http://localhost:7777/document/12345/}
	\end{block}
	\begin{block}{Method}
		\begin{tabular}{lll}
			& \textbf{Element} & \textbf{Collection} \\
			GET & List & Retreive \\
			POST & - & Create \\
			PUT & Replace & Update \\
			DELETE & Delete & Delete \\
		\end{tabular}
	\end{block}
\end{frame}

\begin{frame}
	\frametitle{\insertsection}
	\framesubtitle{Jersey}
	\begin{block}{Main}
		HttpServer
	\end{block}
	\begin{semiverbatim}
		Create HTTP server
	\end{semiverbatim}
	\begin{block}{Create}
		\begin{itemize}
			\item @Path
			\item @POST
			\item @FormParam
		\end{itemize}
	\end{block}
	\begin{semiverbatim}
		Create collection class
	\end{semiverbatim}
\end{frame}

\begin{frame}
	\frametitle{\insertsection}
	\framesubtitle{Jersey}
	\begin{block}{List}
		\begin{itemize}
			\item @GET
			\item @Produces $\rightharpoonup$ media type
		\end{itemize}
	\end{block}
	\begin{semiverbatim}
		Add collection list method
	\end{semiverbatim}
	\begin{block}{Retreive}
		\begin{itemize}
			\item @GET
			\item @PathParam
			\item @Produces $\rightharpoonup$ media type
		\end{itemize}
	\end{block}
	\begin{semiverbatim}
		Create resource class
	\end{semiverbatim}
\end{frame}

%%%%% REFERENCES %%%%%%%%%%%%%%%%%%%%%%%%%%%%%%%%%%%%%%%%%%%%%%%%%%%%%%%%%%%%%%
\section{References}

\begin{frame}
	\frametitle{\insertsection}
	XML\\
	\begin{itemize}
		\item \url{http://www.sitepoint.com/xml-namespaces-explained/}{}
	\end{itemize}
\end{frame}

\begin{frame}
	\frametitle{\insertsection}
	XSD\\
	\begin{itemize}
		\item \url{http://www.w3.org/XML/Schema.html\#dev}{}
		\item \url{http://en.wikipedia.org/wiki/XML_Schema_(W3C)}{}
	\end{itemize}
\end{frame}

\begin{frame}
	\frametitle{\insertsection}
	XSLT\\
	\begin{itemize}
		\item \url{http://en.wikipedia.org/wiki/XSLT_elements}{}
		\item \url{http://www.xsltfunctions.com/}{}
		\item \url{http://incrementaldevelopment.com/xsltrick/}{}
		\item \url{http://saxon.sourceforge.net/\#F9.4HE}{}
	\end{itemize}
\end{frame}

\begin{frame}
	\frametitle{\insertsection}
	JAXB\\
	\begin{itemize}
		\item \url{http://jaxb.java.net/guide/index.html}
		\item \url{http://stackoverflow.com/questions/1514110/}
		\item \url{http://wiki.eclipse.org/EclipseLink/Examples/MOXy/JAXB/SpecifyRuntime}
	\end{itemize}
\end{frame}

\begin{frame}
	\frametitle{\insertsection}
	REST\\
	\begin{itemize}
		\item \url{http://en.wikipedia.org/wiki/Representational_state_transfer}
		\item \url{http://www.ics.uci.edu/~fielding/pubs/dissertation/rest_arch_style.htm}
		\item \url{http://jersey.java.net/nonav/documentation/latest/user-guide.html}
		\item \url{http://www.vogella.com/articles/REST/article.html}
		\item \url{http://www.nakov.com/blog/2009/07/15/jax-rs-path-pathparam-and-optional-parameters/}
		\item \url{http://engineering.linkedin.com/architecture/restli-restful-service-architecture-scale}
	\end{itemize}
\end{frame}

\begin{frame}
	\frametitle{\insertsection}
	JSON\\
	\begin{itemize}
		\item \url{http://json.org/}
		\item \url{http://avro.apache.org/docs/current/spec.html}
		\item \url{https://github.com/linkedin/rest.li/wiki/DATA-Data-Schema-and-Templates}
	\end{itemize}
\end{frame}

\begin{frame}
	\frametitle{\insertsection}
	\LaTeX\\
	\begin{itemize}
		\item \url{http://ftp.snt.utwente.nl/pub/software/tex/macros/latex/contrib/beamer/doc/beameruserguide.pdf}
		\item \url{http://texlipse.sourceforge.net/}
		\item \url{http://dev.cdhq.de/eclipse/word-wrap/}
	\end{itemize}
\end{frame}

\end{document}
